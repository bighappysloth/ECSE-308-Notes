\documentclass[../../main.tex]{subfiles}

\begin{document}
\subsection{Chapter D}
\begin{wtr}[Multiplexing vs. Multiple-access]
    \begin{itemize}
    \item[]
    \item Centralized access control: Multiplexing, using some demand assignment procedure.
    \item MUX: can also use a distributed protocol so that end-systems can ....???
    \item Multiplexing techniques: FDM and TDM. (We will also cover OFDM)
    \end{itemize}
\end{wtr}
\begin{wtr}[FDM]
    \begin{itemize}
        \item[]
        \item Given $N$, data streams. Separate into $N$ sub-channels, each(*) with a non-zero buffer between.
        \item $N$ data streams $\implies N-1$ guard bands.
    \end{itemize}
\end{wtr}

\begin{wtr}[OFDM]

\end{wtr}
\begin{wtr}[TDM]
\begin{itemize}
    \item[]
    \item Require synchronized clocks
    \item Circuit switching
    \item Problem with changing the number of users
\end{itemize}
\end{wtr}

\begin{wtr}[Multiple Acces]
\begin{itemize}
    \item[]
    \item Allow multiple devices to use same medium
    \item FDMA, TDMA CDMA
    \item CDMA - dominant tech for 3g
    $D$ data rate bps,\\
    each user has chipping Compilednew channel has $kD$ chips per second
\end{itemize}
\end{wtr}
\end{document}

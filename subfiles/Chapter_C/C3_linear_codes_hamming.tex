\documentclass[../../main.tex]{subfiles}

\begin{document}
\subsection{C3 Linear Codes}
\begin{definition}
    Any non-empty, subspace $C$ of $F^n$ with dimension $k$ where $1\leq k\neq n$ is called a $(n,k)$ linear code. In particular,\\
    
    If $F+\mathbb{Z}_2$, the $n$-parity check code is a subspace.
\end{definition}

\subsubsection{Introduction to Hamming Weights}
Fix a code $C\subseteq F^n$, where $F^n$ in this section will always denote the finite cartesian product of $F=GF(p)$. Elements in $C$ are called code words. For any $c\in C$, we define its Hamming Weight 
\[
\wt(c) = \biggl|\{i,\,a_i\neq 0\}\biggr|
\]
$\wt(c)$ simply counts the number of non-zero digits. Therefore
\begin{itemize}
    \item For every $x\in C$, $0\leq \wt(x)\leq n$,
    \item If $x\in C$, $\wt(x)=0\iff x=0$ (recall that $C$ is a subspace of $F^n$, so $0\in C$),
    \item $\wt(\cdot)$ induces a metric on $C$, $d(\cdot,\cdot)$ 
    \[
    d(x,y)=\wt(x-y),\quad\forall x,\,y\in C
    \]
\end{itemize}
To verify that the properties of a metric all hold for this $d(x,y)$ on $C$ 
\begin{enumerate}
    \item $d(x,x)=0\iff \wt(x-x)=0\iff x-x=0\iff x=x$,
    \item For every pair of elements $x,y\in C$, $d(x,y)\geq 0$. This comes from the fact that $x-y\in C$ and $\wt(x-y)\geq 0$,
    \item If $v,u,w\in C$ then
    \[
    d(v,w)\leq d(v,u) + d(u,w)
    \]
    This is an easy consequence of 
    \begin{equation}\label{hamming weight triangle inequality}
    \wt(a+b)\leq \wt(a)+\wt(b)
    \end{equation}
    If we write $v-u = a$, and $u-w = b$, then $a+b = v-w$. Indeed, if $a = (a_i)$, $b=(b_i)$, then if for some $1\leq j\leq n$ such that $a_j + b_j\neq 0$ then either $a_j\neq 0$ or $b_j\neq 0$. Therefore Equation $\eqref{hamming weight triangle inequality}$ holds.
\end{enumerate}
Furthermore, define a ball $B_r(w)$ about some $w\in F^n$ as follows
\begin{equation}\label{ball definition in F^n}
B_r(w) = \{x\in F^n,\, d(w,x)\leq r\}
\end{equation}
Notice how this differs from a regular ball in a metric space. We now turn our attention to a quantity which will be of great interest. Fix a $(n,k)$ linear code $C$, the minimum distance $d$ of $C$ is defined to be the smallest distance between two distinct code words in $C$, precisely
\begin{equation}\label{minimum distance code}
    d = \min\{d(v,w),\, v,\,w\in C,\, v\neq w\}
\end{equation}
This immediately implies that $d\geq 1$. If our code contains only one element (namely if $C$ is the trivial subspace), then $d = + \infty$.\\

\begin{definition} A code $C$ uses the Nearest Neighbour Decoding if $w$ is a received word, and if $w\notin C$, then $w$ is decoded as the code word in $C$ that is the closest to it with respect to $d(\cdot,\cdot)$. If $w\in C$ then it is decoded as $w$. If there is a tie, then the nearest neighbour is chosen arbitrarily.
\end{definition}
\begin{wts}
    If $C$ is an $n$-code with minimum distance $d$, (meaning that it does not have to be a subspace). If $C$ uses Nearest Neighbour Decoding, and if $t = d(w,c)$, there $w$ and $c$ are the received and transmitted words, then
    \begin{itemize}
        \item If $t<d$, then $C$ can detect $t$ errors,
        \item If $2t<d$, then $C$ can correct $t$ errors.
    \end{itemize}
\end{wts}
\begin{proof}
    Let $c\in C$ and the received code $w$ satisfies $d(w,c)=t<d$, so the received code is $t$ distance away from the original code. Since $w\notin C$ by definition of $d$ (if $w\in C$ then that would mean $d\leq t$). $C$ can detect such an error, and we say that $C$ can correct $t<d$ errors.\\
    
    If $2t<d$, then $w$ lies only within the $t$-ball of $c$. This is because the $t$-balls about the code words in $C$ are pairwise disjoint. Indeed, if $w\in B_t(c)\cap B_t(c')$ for another code word $c'\in C$, then
    \[
    d\leq d(c',c)\leq d(w,c) + d(w,c')\leq 2t
    \]
    Leading to a contradiction.
\end{proof}
We are ready to prove the main theorems in this Chapter concerning Linear Codes.
\begin{wts}
    Let $C$ be an $(n,k)$-code that uses Nearest Neighbour Decoding, and $d$ be the same as above. Then
    \begin{enumerate}
        \item $d = \min\{\wt(w),\, w\in C,\, w\neq 0\}$,
        \item $C$ can detect $t\geq 1$ errors if and only if $t<d$,
        \item $C$ can correct $t\geq 1$ errors if and only if $2t<d$,
        \item If $C$ can correct $t\geq 1$ errors, and denote $|F| = q$ (if $F =\mathbb{Z}_2$, then $|F|=2$), then
        \[
        \binom{n}{0} + \binom{n}{1}(q-1)+\binom{n}{2}(q-1)^2 + \cdots + \binom{n}{t}(q-1)^t \leq q^{n-k}
        \]
    \end{enumerate}
\end{wts}
\begin{proof}
    We skip the proof for now.
\end{proof}

\begin{definition}
    The coding rate for a $(n,k)$-linear code is $k/n$.
\end{definition}


\begin{definition}
    A binary linear code is called MDS (maximum distance-separable) code if and only if $d$ is equal to its upper bound. (Also called Perfect Codes).
    \[
    d = \{d(x,y),\,x,y\in C,\, x\neq y\} = (n-k) + 1
    \]
\end{definition}

\end{document}
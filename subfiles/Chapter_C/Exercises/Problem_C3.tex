\documentclass[../../../main.tex]{subfiles}

\begin{document}
\subsection{Problem C3}
\begin{wts}
    Using the $16$-bit Internet Checksum:
    \begin{enumalpha}
        \item Calculate the Internet checksum for the four $16$-bit words of data as follows: E308 E309 00FF 0F0F.
        \item Suppose that the received data and checksum are contaminated with error as follows: E308 E309 00FF 0F0E 29DF. Verify that the internet checksum can detect the error.
        \item Find an example of a 2-bit error that can fool Internet checksum for the above data and checksum.
    \end{enumalpha}
\end{wts}
\begin{proof}
Part a. Procedure for calculating the internet checksum:
\begin{enumerate}
    \item Using HEX mode on Calculator, add all the data bits together. 
    \[
    E308+E309+00FF+0F0F = 0001D61F
    \]
    \item Since there is a carry-out of '1', we subtract $10000$ and add $1$, equivalently:
    \[
    0001D61F - 10000 + 1 = D620
    \]
    \item Taking the one's complement, (bitwise XOR) of the sum. This is equivalent to subtracting with FFFF, therefore
    \[
    FFFF- D620 = 29DF
    \]
\end{enumerate}
Part b. Verify that the internet checksum detects the error. The procedure is as follows
\begin{enumerate}
    \item Using HEX mode on Calculator, add all the shit together again,
    \[
    E308+E309+00FF+0F0E+29DF = 0001FFFD
    \]
    \item Carry-out of '1', we need to subtract 10000 and add 1, hence
    \[
    0001FFFD - 10000 + 1 = FFFE
    \]
    \item Take one's complement. Which equates to
    \[
    FFFF-FFFE = 0001 \neq 0000
    \]
    \item Therefore the internet checksum detects the error.
\end{enumerate}
Part c. We note that $E309$ and $E308$ are one bit different from each other, therefore we can apply the error which swaps $9$ with $8$. Resulting in E309 E308 00FF 0F0F 29DF.

\end{proof}

\end{document}